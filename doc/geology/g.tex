\documentclass[12pt]{article}

\usepackage{fontspec}
\usepackage{polyglossia}
\usepackage{unicode-math}

\setmainfont{Times New Roman}
\setmathfont{Cambria Math}

\title{$\mathcal{L}$agrangian Geology module \\ Bad docs}

\begin{document}
    \maketitle

    Fully custom generation support.

    Just one step of gen is not enough.

    3D biome many ids system

    Rocks

    Can generate rocks on predefined and custom generations.  

    Collided or not (Giant impact hypothesis, true - igneous rocks, false - metamorphic dominant).

    World divided by tectonic plates (getted post-factum (tricky) in predefined generations).

    Divided by areas of rocks (orogen, shield, yield etc)

    Dominant elements is oxygen and silicon in vanilla generation.

    Main types of rock

    Igneous {intrusive, extrusive}

    Sedimentary {Clastic, Chemical, Clay, Other}

    Metamorphic {meta if known}{texture: Schists, Gneisses, Granofels}{faces: Eclogite, Blueschist etc}
    {sedimentary para or igneous meta}
    A hornfels is a granofels that is known to result from contact metamorphism.

    //Dominant minerals ionic.

    trace and antitrace element

    Continental
    Oceanic

    Islands
    Mounts
    Volcano


    World layers

    sediments, evaporite

    \begin{enumerate}
        \item Regolith (soil)
        \item Bedrock layers
        \item "Bedrock" and Void (instead mantle - too small height)
    \end{enumerate}

    Bedrock parts

    \begin{enumerate}
        \item Country rocks - oxides, silicates, carbonates, aluminates (?) etc - compounds of most abundant elements.
        \item Igneous intrusives, rare/valuable rocks (pipes, veins, VMS, BIF, skarn etc) on country rocks.
    \end{enumerate}
\end{document}