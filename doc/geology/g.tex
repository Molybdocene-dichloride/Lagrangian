\documentclass[12pt]{article}

\usepackage{fontspec}
\usepackage{polyglossia}
\usepackage{unicode-math}

\setmainfont{Times New Roman}
\setmathfont{Cambria Math}

\title{$\mathcal{L}$agrangian Geology module \\ Ref}

\begin{document}
    \maketitle

    Fully custom generation support.

    Pos lower 0 and above 256 not supported (extremely hard and complex to implement).

    Just one step of generation is not enough.

    Flexible generation API (Abstraction layer on top of Feature and Biome)

    GenFeature - a zone of crust, with 3D complex form, 3D position and many IDs support
    For example Oceans, Orogens etc.

    Biome - biological purpose GenFeature.
    For example Deadlands (by default), Microb mats, Boreal forests etc.

    pre GenFeature -> Biome -> post GenFeature

    MapChunk structure 2x2x2 to 8x8x8 blocks("chunks") (32 to 128 nodes("blocks")) used only for generation.

    MapGen interface with makeChunk(BlockMake) method for generation.

    World divided by tectonic plates (getted post-factum (tricky) in predefined generations).

    Rocks

    Can generate rocks on predefined and custom generations.  

    Giant impact hypothesis (true - igneous rocks, false - metamorphic dominant).

    Dominant elements is oxygen and silicon in vanilla generation.

    Main types of rocks

    \begin{itemize}
		\item Igneous {intrusive, extrusive}
        \item Sedimentary {Clastic, Chemical, Clay, Other}{sedimentary para or igneous meta}
        \item Metamorphic {meta if known}{texture: Schists, Gneisses, Granofels}
    \end{itemize}
    
    //Dominant minerals ionic.

    Trace and antitrace element

    Crust types

    \begin{itemize}
		\item Continental
        \item Subcontinental
        \item Suboceanic
        \item Oceanic
    \end{itemize}

    Volcano

    World layers

    sediments, evaporite

    \begin{enumerate}
        \item Regolith (soil)
        \item Bedrock layers
        \item "Bedrock" and Void (instead of a mantle, as the height is too small)
    \end{enumerate}

    Bedrock parts

    \begin{enumerate}
        \item Country rocks - oxides, silicates, carbonates, aluminates (?) etc - compounds of most abundant elements.
        \item Igneous intrusives, rare/valuable rocks (pipes, veins, VMS, BIF, skarn etc) on country rocks.
    \end{enumerate}
\end{document}